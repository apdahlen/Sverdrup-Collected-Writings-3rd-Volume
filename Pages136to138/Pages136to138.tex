
\begin{center}
\includegraphics[width=0.9\textwidth]{OpenImage.png}
\end{center}

\section{The Lord Exalteth the Lowly}

Manuscript. No year is indicated. Yet the handwriting suggests that the discourse is from the beginning or middle of the eighties. The title has been added by Wg.

This section appears on pages 354–361 of the original volume. — Present Ed.

\bigskip

The Reformation is one of those remarkable upheavals which the human race experiences only very few of. The old is torn up and overthrown; that which was great is set beneath, and that which was little is set above. Of such an upheaval Hannah, the mother of Samuel, sang when she brought her son up to the temple of the Lord; of such an upheaval Mary sang, when she was the mother of Jesus:

\begin{quote}
“The Lord maketh poor, and maketh rich;\\ 
He bringeth low and lifteth up also;\\ 
He raiseth up the poor out of the dust,\\ 
and lifteth up the beggar from the dunghill,\\ 
to set them among princes,\\ 
and to make them inherit the throne of glory.”
\end{quote}

Such an upheaval took place in the highest sense in the life and work of Jesus, when the empire of Caesar fell and the Kingdom of God was planted upon its ruins.

But a similar upheaval took place also at the Reformation. The congregation, trodden down and scorned, was lifted up by Luther unto freedom and restored to its rights. Yet the true vital force in this work, which has transformed the history of Europe and of the world, was that little secret which is in truth the great upheaval in a human heart, whereby that which is great is cast down and that which is little is lifted up. It is a Reformation in the small; it is a thought of eternity laid down within the inward realm of thought. Eternal life in the land of death.

For with faith it is thus: it lays a human being so low as he can possibly come, in order to lift him up again unto the bright and blessed heights of heaven.

It is an solemn hour when the Word—the living, sharp, two‑edged Word of God—enters a human life and lays the foundation bare in the heart. As a ploughshare breaks up the earth and turns it over, so the Word of the Lord ploughs in the soil of the heart. Then a human being becomes so small and poor, and God exalts Himself in human hearts, until He fills all things with His holy presence. Holy, holy, holy is the Lord of Hosts, O sinner! Who then art thou, that thou shouldest stand before His face?

Then a human heart is crushed down into unquenchable anguish; and great is the Lord alone in that day. It is the Day of Judgment in a sinner’s heart. Then its pride and strength are cast down; then all its work becomes insipid and vanity; then its best righteousness has become a stinking abomination; then its hopes sink like withered flowers and as dry grass, for the breath of the Lord hath blown upon them.

How exceedingly small it has then become, how feeble and broken its strength! Great words and cheerful smiles avail nothing; it is altogether undone, and in terror a worm shrinks upon the ground before the holy God.

But when then the Holy One bends Himself down to the poor sinner and lifts him up unto Himself and says, Thou art Mine; I have redeemed thee—then is he in truth set upon the throne of glory.

When the clenched hand, by which a human being clung to the world’s mere straws, is unclenched, and the soul lays hold on its Jesus as rescue from the threatening doom; and when it perceives that instead of threatening blows it is tender, loving arms that receive it—then there grows within the soul a full trust, a firm assurance, an unshakable faith, which bears through life and death.

And when a sinner goes forth again from such a meeting with God, the living God, then he is wholly renewed. God has become so great, and the world so small; God has become so strong, and sin and death so powerless, that all things are measured with a new measure, and confidence and courage spring forth where before there was weakness and fear.

The assurance of grace, that we are the children of God—that is faith.
And this is the secret of the Reformation.

For then a human being stands alone with God. The world is gone, sin is gone, death and hell are gone.

Then, even in the night of death, a human being can have the boldness to say: If God be for me, who can be against me?

This is a human being’s Reformation. It casts him down and exalts him; it annihilates him and makes him alive.

Blessed is he who was thus terrified, since he was thus raised from the dead.

These are the children of the Reformation, who in truth have the life of faith in their hearts, and who from being dead have become living, and who know by their own experience that the Lord exalts the lowly.
