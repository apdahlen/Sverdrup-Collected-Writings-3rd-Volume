Preface

As mentioned in the subscription invitation issued more than a year ago, the intention was that one volume of Professor Sverdrup’s Collected Writings in Selection should deal with “the education of pastors, and this especially in connection with the school in which the author carried out his long and blessed work.” As the work of organizing the exceedingly rich material progressed, it became clear, however, that a considerable portion of what was originally intended to be included in this volume would almost necessarily have to find its place in the volume dealing with the congregation, if a break in the coherence of several of the articles published there was to be avoided. The author also regarded the education of pastors as an exceedingly important link in the work of “building the congregation.”

This made it necessary to alter the plan for the present volume somewhat, with the result that it deals chiefly with Augsburg Seminary and Lutheran church doctrine. As such, it is for the most part of a historical nature, since I have included excerpts from the reports that the author submitted in connection with his work for the school and the church. With reference to what is stated below, page 29, note, concerning the omissions that have been made, I will say here only that it has been my intention and is my hope that the published statements from the reports will be suitable and sufficient to give the general reader a reasonably clear picture of the institutions concerned, whereas the historically and scientifically trained student will in any case have to turn to the original sources.

Here, too, I have sought to follow the principles for the treatment of the material that I set forth in the preface to the first and second volumes. Many persons and circumstances had at the time necessarily to be mentioned in reports and elsewhere, in order that the situation might become clear and unmistakable to those who were affected by it in one way or another. When several names have been omitted here, although they will surely have to find a place again in a future North American church history, this has been done in the recognition that no one would be readier than the author himself to concede that “one should write feelings in sand, but benefactions in metal.”

As is well known, Professor Sverdrup died early in the morning of May 3, 1907. He had been at the school on Tuesday, but not on Wednesday and Thursday. On Thursday evening the school’s closing celebration was held, conducted by Professor Stetzel in Veitkirch’s stead. It will also be remembered that the “Augsburg editor,” Pastor Ole Paulson, had died a little more than a week earlier. After Professor Sverdrup’s death, a piece of paper written in pencil was found on his desk. From its contents it was easy to see that it was a preliminary draft of a speech for the closing celebration. In a few short sentences it contains, as is said, the whole of Professor Sverdrup’s outlook on life. In this volume, where a series of speeches delivered by him on various occasions has been gathered, it seems fitting that this, his last speech—the one he never came to deliver—should also be included. There is reason to believe that through these few, simple yet soulful words he still speaks, though he is dead.

With the hope that the present volume will, for many, revive old, valuable memories and help toward a better understanding and a more just assessment of Augsburg Seminary and Lutheran church doctrine, it is sent out on its way.

Augsburg Seminary, May 7, 1910.

Andreas Helland.